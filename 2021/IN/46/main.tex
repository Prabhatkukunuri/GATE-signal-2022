\iffalse
\let\negmedspace\undefined
\let\negthickspace\undefined
\documentclass[journal,12pt,twocolumn]{IEEEtran}
\usepackage{cite}
\usepackage{amsmath,amssymb,amsfonts,amsthm}
\usepackage{algorithmic}
\usepackage{graphicx}
\usepackage{textcomp}
\usepackage{xcolor}
\usepackage{txfonts}
\usepackage{listings}
\usepackage{enumitem}
\usepackage{mathtools}
\usepackage{gensymb}
\usepackage{comment}
\usepackage[breaklinks=true]{hyperref}
\usepackage{tkz-euclide} 
\usepackage{listings}
\usepackage{gvv}                                        
\def\inputGnumericTable{}                                 
\usepackage[latin1]{inputenc}                                
\usepackage{color}                                            
\usepackage{array}                                            
\usepackage{longtable}                                       
\usepackage{calc}                                             
\usepackage{multirow}                                         
\usepackage{hhline}                                           
\usepackage{ifthen}                                           
\usepackage{lscape}
\newtheorem{theorem}{Theorem}[section]
\newtheorem{problem}{Problem}
\newtheorem{proposition}{Proposition}[section]
\newtheorem{lemma}{Lemma}[section]
\newtheorem{corollary}[theorem]{Corollary}
\newtheorem{example}{Example}[section]
\newtheorem{definition}[problem]{Definition}
\newcommand{\BEQA}{\begin{eqnarray}}
\newcommand{\EEQA}{\end{eqnarray}}
\newcommand{\define}{\stackrel{\triangle}{=}}
\theoremstyle{remark}
\usepackage{circuitikz}
\newtheorem{rem}{Remark}
\begin{document}
\parindent 0px

\bibliographystyle{IEEEtran}
\vspace{3cm}

\title{Assignment\\[1ex]GATE-IN-46}
\author{EE23BTECH11034 - Prabhat Kukunuri$^{}$% <-this % stops a space
}
\maketitle
\newpage
\bigskip

\renewcommand{\thefigure}{\theenumi}
\renewcommand{\thetable}{\theenumi}
\section{Question}
Consider a system with transfer-function $G\brak{s}=\frac{2}{s+1}$. A unit-step function $\mu\brak{t}$ is applied to the system, which results in an output y\brak{t}. 

If $e\brak{t}=y\brak{t}-\mu\brak{t}$ then $ \lim_{t\to\infty} e(t)$ is\rule{1.5cm}{0.15mm}.

\solution
\fi
\begin{table}[h]
    \centering
    \begin{tabular}{|p{1cm}|p{3.00cm}|p{3.50cm}|}
    \hline
    Symbol&Value&Description\\ \hline
    $$G\brak{s}$$&$$\frac{2}{s+1}$$&$$\text{Transfer function}$$\\\hline
    $$e\brak{t}$$&$$y\brak{t}-\mu\brak{t}$$&$$\text{Function of y\brak{t} and $\mu\brak{t}$}$$\\\hline
    $$Y\brak{s}$$&$$G\brak{s}\times U\brak{s}$$&Convolution in $t$ domain is multiplication in $s$ domain.\\\hline
    $$\mu\brak{t}$$&$$\begin{cases}
    0&\text{if t$<$0}\\
    1&\text{if t$>$0}
    \end{cases}$$&$$\text{Unit step function}$$\\\hline
    \end{tabular}
    \caption{Variable description}
    \label{tab:GATE.2021.IN.46.1}
\end{table}\\
Applying Laplace transform on $\mu\brak{t}$
\begin{align}
    &\mu\brak{t}\system{L}U\brak{s}\\
    &U\brak{s}=\frac{1}{s}\\
    &Y\brak{s}=\brak{\frac{2}{s+1}}\brak{\frac{1}{s}}\\
    &Y\brak{s}=\frac{2}{s}-\frac{2}{s+1}
\end{align}
The inverse Laplace transform of $\frac{a}{s+b}$ is $ae^{-bt}\mu\brak{t}$
\begin{align}
    &y\brak{t}=2\mu\brak{t}-2e^{-t}\mu\brak{t}\\
    &e\brak{t}=\mu\brak{t}\brak{1-2e^{-t}}\\
    &\lim_{t\to\infty}e\brak{t}=\lim_{t\to\infty}\mu\brak{t}\brak{1-2e^{-t}}\\
    &\lim_{t\to\infty}e\brak{t}=1
\end{align}
%\end{document}
